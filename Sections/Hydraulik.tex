\section{Fluiddynamik}
\subsection{Auftriebskraft \kuchling{154}}
\begin{formulaexpanded}
	{F_A = V_{Fl} \cdot \rho_{Fl} \cdot g = m_{Fl} \cdot g}
	\rho & Dichte & [\frac{kg}{m^3}] \\
	V_{Fl} & Volumen verdrängten Flüssigkeit & [m^3] \\
\end{formulaexpanded}

\subsection{Oberflächenspannung}
\begin{formulaexpanded}
	{\sigma = \frac{F}{l}}
	\sigma & Grensflächenspannung & [] \\
	F & Kraft & [N] \\
	l & Länge & [m] \\
\end{formulaexpanded}

\begin{formulaexpanded}
	{h = \frac{2\sigma}{\rho g r} = \frac{\sigma}{\rho g d}}
	\sigma & Grensflächenspannung & [] \\
	h & Kapillarität \kuchling{179} & [N] \\
	\rho & Dichte & [\frac{kg}{m^3}] \\
	g & Erdbeschl. & [\frac{m}{s^2}] \\
	r & Radius der Kapillare & [m] \\
	d & Durchmesser der Kapillare & [m] \\
\end{formulaexpanded}

\begin{formulaexpanded}
	{p = \frac{2 \sigma}{r}}
	\sigma & Grensflächenspannung & [] \\
	r & Radius der Kapillare & [m] \\
	p & Druck in Kapillare & [Pa] \\
\end{formulaexpanded}

\subsection{Kontinuitäts-/ Bernoulligleichung}
Oft werden folgende Gleichungen für eine Aufgabe benötigt. \kuchling{162ff}
\begin{formulaexpanded}
	{A_1 \cdot v_1 = A_2 \cdot v_2 = \dot V}
	A & Querschnitt an Stelle $x$ & [m^2] \\
	v & Strömungsgeschw. an Stelle $x$ & [m/s] \\
	\dot V & Volumenstrom & [\frac{m^3}{s}] \\
\end{formulaexpanded}

\begin{formulaexpanded}
	{p_1 + \rho g h_1 + \frac{\rho}{2}v_1^2 = p_2 + \rho g h_2 + \frac{\rho}{2}v_2^2}
	p & Druck an Stelle $x$ & [Pa] \\
	h & Höhe an Stelle $x$ & [m/s] \\
	\rho & Dichte der Flüssigkeit & [\frac{kg}{m^3}] \\
	v & Strömungsgeschw. an Stelle $x$ & [m/s] \\
\end{formulaexpanded}

\subsection{Strömungen}
Dynamische Viskosität $\eta$ \kuchling{168}
\begin{formulaexpanded}
	{\tau = \eta \frac{v}{d}}
	\tau & Schubspannung & [F] \\
	\eta & Zähigkeit & [Pa\cdot s] \\
	d & Abstand der Begrenzungsflächen voneinander & [m] \\
	v & Relativgeschwindigkeit zwischen Begrenzungsflächen & [m/s] \\
\end{formulaexpanded}

\begin{minipage}{\textwidth}		
	\begin{minipage}{0.2\textwidth}
		\includegraphics[width=\columnwidth]{Images/Strömungen}
	\end{minipage}%%% to prevent a space
	\begin{minipage}{0.3\textwidth}
		\begin{enumerate}[nosep]
			\item Laminar Rechnen
			\item Aus Resultat $Re$ berechnen
			\item mit $Re_{kritisch}$ vergleichen
			\item bei überschreiten Turbulent rechnen
		\end{enumerate}
	\begin{formulaexpanded}
		{\text{Lam:}	\lambda = \frac{64}{Re}  \quad \text{Turb:} \lambda = \frac{0.316}{\sqrt[4]{Re}} \\
		\Delta p = \lambda \frac{l\rho v^2}{2 d}}
		\Delta p & Druckabfall & [Pa] \\
		l & Rohrlänge & [m] \\
		d & Rohrdurchmesser & [m] \\
		\rho & Dichte des Fluids & [\frac{kg}{m^3}] \\
		v & Fliessgeschw. & [\frac{m}{s}]
	\end{formulaexpanded}
	\end{minipage}
\end{minipage}

\begin{minipage}{\textwidth}	
	\kuchling{173}~\\

	\begin{minipage}{0.12\textwidth}
		\includegraphics[width=\columnwidth]{Images/flugzeugflügel}
		\\~\\~\\~\\~\\
		\\~\\~\\~\\~\\
		\includegraphics[width=\columnwidth]{Images/flugzeugflügel1}
	\end{minipage}%%% to prevent a space
	\begin{minipage}{0.35\textwidth}
		\begin{formulaexpanded}
			{F_D = c_w \cdot A_s \cdot \frac{\rho}{2} \cdot v^2}
		\end{formulaexpanded}
		\begin{formulaexpanded}
			{F_A = F_G = c_a \cdot A_p \cdot \frac{\rho}{2} \cdot v^2}
		\end{formulaexpanded}	
		\begin{formulaexpanded}
			{F_V = F_W = c_w \cdot A_p \cdot \frac{\rho}{2} \cdot v^2}
			F_x & Widerstand & [F] \\
			F_D & Druckwiderstand $\neq$ $F_V$ & [F] \\
			c_w & Widerstandszahl & [1] \\
			\rho & Dichte & [\frac{kg}{m^3}] \\
			v & Strömungsgeschwindigkeit & [m/s] \\
			A_p & projezierte Fläche $\parallel$ Strömung & [m^2] \\
			A_s & projezierte Fläche $\perp$ Strömung & [m^2] \\
		\end{formulaexpanded}
	~\\
		\begin{formulaexpanded}
			{\tan \varphi = \frac{F_W}{F_A} = \frac{c_w}{c_a} = \frac{v_v}{v_h}}
			\varphi & Winkel & [\circ] \\
			c_x & Widerstands/-Auftriebskoeff. & [1] \\
			v_x & Vert./-Hori. Geschwindigkeit & [m/s] \\
		\end{formulaexpanded}
	\end{minipage}
\end{minipage}
