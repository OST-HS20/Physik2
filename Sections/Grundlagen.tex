\section{Grundlagen}
\subsection{Einheiten}
\begin{formula}
	{p = \frac{F}{A}}
	F & Kraft (Senkrecht zur Fläche $A$) & [N] \\
	A & Fläche & [m^2] \\
	p & Druck & [Pa] \\
\end{formula}

\begin{formula}
	{\rho = \frac{m}{V} = \frac{p\cdot m}{R \cdot T}}
	\rho & Dichte & [\frac{kg}{m^3}] \\
	m & Masse & [kg] \\
	M & Mol-Masse & [mol] \\	
	V & Volumen & [m^3] \\
	p & Druck & [Pa] \\
	T & Temperatur & [K] \\	
\end{formula}

\subsection{Umrechnungen}
\begin{formula}
	{F = C \cdot \frac{9}{5}+32}
	F & Fahreinheit & [F] \\
	C & Celsius & [C] \\
\end{formula}
\begin{formula}
	{K = C + 273.15}
	K & Kelvin & [K] \\
	C & Celsius & [C] \\
\end{formula}

\noindent Umrechnung von Druck-Einheiten auf \kuchling{151}. 
\begin{formula}
	{1bar = 10^5Pa}
\end{formula}
\begin{formula}
	{1l = 1dm^3}
\end{formula}
