\section{Thermodynamik}
\subsection{Konstanten}
Avogadro-Konstante: $N_A = 6.022\cdot10^{23} \frac{1}{mol}$\\
Bolzmann-Konstante: $k_B = 1.381 \cdot 10^{-23} \frac{J}{K}$\\
Gas-Konstante: $R = k_B \cdot N_A = 8.314 \frac{J}{mol\cdot K}$\\

\begin{formula}
	{p \cdot V = N \cdot k_B \cdot T}
	N & Anzahl Molekühle & [1] \\
	T & Temperatur & [K] \\
\end{formula}

\subsection{Ausdehnungen \kuchling{244}}
\noindent\begin{formula}
	{\Delta l = \alpha l \Delta T}
\end{formula}
\noindent\begin{formula}
	{\Delta A = \beta A \Delta T \qquad \beta = 2\alpha}
	l & Original-Länge & [m] \\
	A & Original-Fläche & [m^2] \\
	V & Original-Volumen & [m^3] \\
	\Delta T & Temp. Diff & [^\circ C]
\end{formula}
\noindent\begin{formula}
	{\Delta V = \gamma V \Delta T \qquad \gamma = 3\alpha}
\end{formula}

\begin{formula}
	{\sigma = E \cdot \alpha \cdot \Delta T}
	\sigma & Thermische Spannung & [] \\
	E & Elastizitätsmodul & [] \\
	\alpha & Längenausdehnungskoeff. & [] \\
	\Delta T & Temp. Diff & [^\circ C]
\end{formula}

\subsection{Gase}
\subsubsection{Ideale Gase}
\textbf{Ideales Gas} kann verwendet werden wenn Druck/Temp. oder Volumen weit vom \underline{kritischen Punkt} entfernt sind. Dabei sind die Gasteilchen masselos und frei von Anziehungs/Abstossungskräften.
\begin{formula}
	{n = \frac{m}{M} = \frac{N}{N_A}}
	n & Stoffmenge & [mol] \\
	m & Masse & [kg] \\
	M & Mol-Masse (Periodesys.) & [mol] \\
	N & Anzahl Zeilchen & [1]
\end{formula}
\begin{formula}
	{p\cdot V = n \cdot R \cdot T}
	p & Druck & [Pa] \\
	V & Volumen & [m^3] \\
	n & Stoffmenge & [mol] \\
	T & Temp & [K]
\end{formula}



\subsubsection{Zustandsänderung \kuchling{297}}
Ein Kreisprozess ist die umschlossene Fläche des P-V-Diagramms.\\
\noindent\textbf{Isochor} (gleiches Volumen) \kuchling{287}
\begin{formula}
	{\frac{p}{T} = Konstant}
\end{formula}
\begin{formula}
	{Q = m \cdot c_{v} \cdot \Delta T}
	Q & Wärmeenergie & [J] \\
	c_v & spezifische Wärmekapazität & [\frac{J}{kgK}] \\
	m & Masse des Gases & [kg] \\
\end{formula}
\begin{formula}
	{W = 0}
	W & Arbeit & [J] \\
\end{formula}
~\\

\noindent\textbf{Isobar}  (gleicher Druck)\kuchling{288}
\begin{formula}
	{\frac{V}{T} = Konstant}
\end{formula}
\begin{formula}
	{Q = m \cdot c_{p} \cdot \Delta T}
	Q & Wärmeenergie & [J] \\
	c_p & spezifische Wärmekapazität & [\frac{J}{kgK}] \\
	m & Masse des Gases & [kg] \\
\end{formula}
\begin{formula}
	{W = - p \cdot \Delta V}
	W & Arbeit & [J] \\
	\Delta V & Volumenänderung & [m^3] \\
\end{formula}
~\\

\noindent\textbf{Isotherm} (gleiche Temp.) \kuchling{289}
\begin{formula}
	{p \cdot V = Konstant}
\end{formula}
\begin{formula}
	{Q = W}
	Q & Wärmeenergie & [J]
\end{formula}
\begin{formula}
	{W &= p\cdot V \ln\left(\frac{V_1}{V_2}\right) \\
		&= p\cdot V \ln\left(\frac{p_2}{p_1}\right)}
	W & Arbeit & [J] \\
	V & Volumen & [m^3] \\
	V_1 & Anfangsvolumen & [m^3] \\
	V_2 & Endvolumen & [m^3] \\
	p_1 & Anfangsdruck & [Pa] \\
	p_2 & Enddruck & [Pa] \\
\end{formula}
~\\

\noindent\textbf{Adiabatisch} (Kein Wärmeaustausch) \kuchling{291}
\begin{formula}
	{Q = 0}
	Q & Wärmeenergie & [J]
\end{formula}
\begin{formula}
	{W = c_v \cdot m \cdot \Delta T}
	W & Arbeit & [J] \\
	c_v & spezifische Wärmekapazität & [\frac{J}{kgK}] \\
	m & Masse des Gases & [kg] \\  
\end{formula}
~\\

\subsubsection{Reale Gase \kuchling{281}}

\begin{formula}
	{a = \frac{9}{8}R\cdot T_k \cdot V_{mk}}
\end{formula}
\begin{formula}
	{b = \frac{V_{mk}}{3}}
\end{formula}
\begin{formulaexpanded}
	{\left(p + \frac{n^2 \cdot a}{V^2} \right) \cdot (V - n\cdot b) = n \cdot R \cdot T}
	a,b & VAN-DER-WAALS Konstanten & [*] \\
	p & Druck & [Pa] \\
	V & Volumen & [m^3] \\
	n & Stoffmenge & [mol] \\
	T & Temp & [K] \\
	T_k & Temp-Kritisch & [K] \\
	V_{mk} & Mol-Volumen-Kritisch & [m^3] \\
\end{formulaexpanded}

\subsection{Arbeit}
\begin{formulaexpanded}
	{\Delta W = p \cdot \Delta V}
	\Delta V & Volumenänderung & [m^3] \\
	\Delta W & Arbeit & [J] \\
	p & Druck & [Pa] \\
\end{formulaexpanded}

\subsection{Wärme}
Wärmebilanz. $+$ für Zugeführte Wärme-Menge, $-$ für abgeführte Wärme-Menge:
\begin{formulaexpanded}
	{\sum_{i=1}^{n}\left(\Delta Q_i + Q_{f,i} + Q_{s,i}\right) = 0}
\end{formulaexpanded}

\begin{formulaexpanded}
	{Q_f = q_f \cdot m \qquad Q_s = q_s \cdot m}
	Q_x & Schmelz-/Verdampfungswärme & [J] \\
	q_x & Spezifische Wärme & [\frac{kJ}{kg}] \\
	m & Masse & [kg]
\end{formulaexpanded}

\begin{formulaexpanded}
	{Q = c \cdot m \cdot \Delta T = C \Delta T}
	Q & Wärmekapazität & [J] \\
	C & absolute Wärmekapazität & [\frac{kJ}{kg}] \\
	c & spezifische Wärmekapazität & [\frac{kJ}{kg}] \\
	m & Masse & [kg] \\
\end{formulaexpanded}

\begin{formulaexpanded}
	{\overline{v} = \sqrt{\frac{8 \cdot k_b \cdot T}{\pi \cdot m}} = \sqrt{\frac{8 \cdot R \cdot T}{\pi \cdot M}}}
	\overline{v} & Mittle Molekühl Geschwindigkeit & [m/s] \\
	k_b & Boltzmann-Konstante \kuchling{310} & [J/K] \\
	T & Absolute Temperatur & [K] \\
	M & Molekühlmasse & [kg/mol] \\
	m & Masse & [kg] \\
\end{formulaexpanded}

\subsection{Wärmestrahlung \kuchling{330}}
\begin{formula}
	{K = \epsilon \sigma T^4 = \frac{R^2}{r^2}\cdot \frac{P}{A} \qquad E = \frac{P}{A}}
\end{formula}
\begin{formulaexpanded}
	{P = \sigma \epsilon A T^4}
	P & Strahlungsleistung & [W] \\
	E & Bestrahlungsstärke & [\frac{W}{m^2}] \\
	K & Emmissionsvermögen & [\frac{W}{m^2}] \\
	A & strahlende Oberfläche & [m^2] \\
	T & Absolute Temperatur & [K] \\
	R & Radius d. bestr. Fläche & [m] \\
	r & Wärmequelleradius & [m] \\
	\epsilon & Emissionsgrad \kuchling{648} & [-] \\
	\sigma = 5.670 \cdot 10^{-8} & Konstante & [\frac{W}{m^2K^4}] 
\end{formulaexpanded}


\subsection{Wärmeleitung}
\begin{formulaexpanded}
	{P = \dot{Q}_W = A \cdot J = A\cdot k \cdot \Delta T}
	\dot{Q}_W & Wärmetransport durch Leitung & [W] \\
	J & Wärmestromdichte & [\frac{W}{m^2}] \\
	A & Fläche & [m^2] \\
	k & Wärmedurchgangskoeff. \kuchling{326} & [\frac{W}{m^2K}] \\
\end{formulaexpanded}
\begin{formulaexpanded}
	{P= \dot{Q}_L = c_p \cdot \rho \cdot \dot{V} \cdot \Delta T}
	\dot{Q}_L & Wärmetransport durch Fluid & [W] \\
	c_p & Spezifische Wärmekapazität & [\frac{kJ}{kgK}] \\
	\dot{V} & Volumenstrom & [\frac{m^3}{s}] \\
	\rho & Dichte von Fluid & [] \\
\end{formulaexpanded}

\begin{formulaexpanded}
	{J &= k(T_i-T_a) \\
       &= \alpha_i(T_i - T_{wi}) \\
       &= \alpha_a(T_{wa} - T_a) \\
       &= \sigma \cdot \epsilon(T_i^4 - T_a^4) \\
    }
	J & Wärmestromdichte & [\frac{W}{m^2}] \\
	\lambda & Wärmeleitfähigkeit \kuchling{644} & [\frac{W}{mK}] \\
	\alpha_i & Wärmeüberg.koeff. Innen \kuchling{646} & [\frac{W}{m^2K}] \\
	\alpha_a & Wärmeüberg.koeff. Aussen \kuchling{646} & [\frac{W}{m^2K}] \\
	T_i & Innentemperatur & [^\circ C] \\
	T_{wi} & Wandtemperatur Innen & [^\circ C] \\
	T_a & Aussentemperatur & [^\circ C] \\
	T_{wa} & Wandtemperatur Aussen & [^\circ C] \\
	k & Wärmedurchgangskoeff. \kuchling{326} & [\frac{W}{m^2K}] \\
	\epsilon & Emissionsgrad \kuchling{648} & [-] \\
	\sigma & Konstante & [\frac{W}{m^2K^4}] \\
\end{formulaexpanded}
\newpage
\subsubsection{Wärmekoeffizient}
\noindent\textbf{Wand} mit mehreren Schichten:
\begin{formulaexpanded}
	{\frac{1}{k} = \frac{1}{\alpha_i} + \sum_{j=1}^{m}{\frac{1}{k_j}} + \sum_{i=1}^{n}{\frac{l_i}{\lambda_i}} + \frac{1}{\alpha_a}}
	\alpha_i & Wärmeübergangskoeff. Innen \kuchling{646} & [\frac{W}{m^2K}] \\
	\alpha_a & Wärmeübergangskoeff. Aussen \kuchling{646} & [\frac{W}{m^2K}] \\
	\lambda & Wärmeleitfähigkeit \kuchling{644} & [\frac{W}{mK}] \\
	l & Wanddicke & [m] \\
	k & Wärmedurchgangskoeff. \kuchling{326} & [\frac{W}{m^2K}] \\
\end{formulaexpanded}

\noindent\textbf{Rohr} mit mehreren Schichten:
\begin{formulaexpanded}
	{\frac{1}{k} = r_a \cdot \left[\frac{1}{\alpha_i r_i} + \sum_{x=1}^{n}{\left(\frac{1}{\lambda_x}\ln\left(\frac{r_{xa}}{r_{xi}}\right)\right)} + \frac{1}{\alpha_a r_a}\right]}
	\alpha_i & Wärmeübergangskoeff. Innen & [\frac{W}{m^2K}] \\
	\alpha_a & Wärmeübergangskoeff. Aussen & [\frac{W}{m^2K}] \\
	r_i & Radius Innen & [m] \\
	r_a & Radius Aussen & [m] \\
	r_{xi} & Radius x-te Ebene Innen & [m] \\
	r_{xa} & Radius x-te Ebene Aussen & [m] \\
	\lambda_x & Wärmeleitfähigkeit \kuchling{644} & [\frac{W}{mK}] \\
\end{formulaexpanded}

\subsection{Luftfeuchtigkeit \kuchling{278}}
Bei \textbf{Wasser} kann der \textbf{Dampfdruck} mit der Magnus-Formel berechnet werden. Für andere Fälle Clausius-Clapeyron-Gleichung verwenden.
\begin{formulaexpanded}
	{p_s(\theta) =\left\{\begin{array}{ll}
			\theta \geq 0 & p_{s0} \cdot 10^{\frac{7.5\cdot\theta}{\theta + 237}} \\
			\theta \leq 0 & p_{s0} \cdot 10^{\frac{9.5\cdot\theta}{\theta + 265.5}}
		\end{array}\right.}
	\theta & Temperatur & [^\circ C] \\
	p_{s0} = 610.7Pa & Dampfdruck bei $0^\circ C$ & [Pa] \\
\end{formulaexpanded}

\begin{formulaexpanded}
	{\theta(p_s) =\left\{\begin{array}{ll}
			p_s \geq p_{s0} & \frac{237\cdot\log\left(\frac{p_s}{6.107}\right)}{7.5 - \log\left(\frac{p_s}{6.107}\right)} \\
			p_s \leq p_{s0} &  \frac{265.5\cdot\log\left(\frac{p_s}{p_{s0}}\right)}{9.5 - \log\left(\frac{p_s}{p_{s0}}\right)}
		\end{array}\right.}
	\theta & Temperatur & [^\circ C] \\
	p_{s0} = 610.7Pa & Dampfdruck bei $0^\circ C$ & [Pa] \\
\end{formulaexpanded}

\noindent Der Taupunkt ist bei Luft mit einer bestimmten Luftfeuchtigkeit diejenige Temperatur, die bei konstantem Druck unterschritten werden muss, damit sich Wasserdampf abscheiden kann:
\begin{formulaexpanded}
	{\theta_D(\theta, f_r) = \frac{237 \cdot \left(\log(f_r) + \frac{7.5 \cdot \theta}{\theta + 237}\right)}{7.5-\left(\log(f_r) + \frac{7.5 \cdot \theta}{\theta + 237}\right)}}
	\theta & Temperatur & [^\circ C] \\
	\theta_D & Taupunkt Temperatur & [^\circ C] \\
	f_r & Relative Luftfeuchtigkeit & [\%] \\
\end{formulaexpanded}

\begin{formulaexpanded}
	{f = \frac{m_w}{V} \qquad f_r = \frac{m_w}{m_s} = \frac{p_D}{p_S}}
	f & Absolute Luftfeuchtigkeit & [-] \\
	m_w & Masse Wasserdampf & [kg] \\
	m_s & Masse Wasserdampf bei Sättigung & [kg] \\
	p_D & Partialdruck Wasserdampf & [Pa] \\
	p_S & Sättigungsdruck Wasserdampf & [Pa] \\
	V & Volumen & [m^3] \\
\end{formulaexpanded}

\begin{formulaexpanded}
	{f_{ri} = \frac{p_s(\theta_a)}{p_s(\theta_i)}\cdot f_{ra}}
	f_{ri} & Relative Innen-Feuchte & [\%] \\
	f_{ra} & Relative Aussen-Feuchte & [\%] \\
	\theta_i & Innen Temperatur & [^\circ C] \\
	\theta_a & Aussen Temperatur & [^\circ C] \\
\end{formulaexpanded}

\subsection{Wirkungsgrad}
Adiabaten-Gleichung:
\begin{formulaexpanded}
	{\kappa = \frac{c_{mp}}{c_{mV}}}
	\kappa & Adiabaten-Exponent & [-] \\
	c_{mp} & Molare Wärmekapazität bei konstantem Druck & [] \\
	c_{mV} & Molare Wärmekapazität bei konstantem Volumen & [] \\
\end{formulaexpanded}
\begin{formulaexpanded}
	{pV^\kappa = T V^{\kappa-1} = T^\kappa p^{1-\kappa} = \text{Konstant}}
	p & Druck & [Pa] \\
	V & Volumen & [m^3] \\
	\kappa & Adiabaten-Exponent & [-]
\end{formulaexpanded}


\subsubsection{Carnot-Wirkungsgrad}
\textbf{Wärmekraftmaschine}, Rechts laufender Kreislauf, Mechanische Arbeit wird erzeugt.
\begin{formula}
	{\eta_C = \frac{\Delta T}{T_{hoch}}}
\end{formula}
\textbf{Wärmepumpe}, Links laufender Kreislauf. Mechanische Arbeit wird benötigt.
\begin{formulaexpanded}
	{\eta_{iC} = \frac{T_{hoch}}{\Delta T}}
	\Delta T & Temperatur Diff. von Warm/Kalt Reservoir & [K] \\
	T_{hoch} & Temperatur Warm-Reservoir & [K]
\end{formulaexpanded}